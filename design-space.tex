\section{Design Space}
\label{sec:dspace}

The narrowly-defined interfaces representing the dominating trend in storage systems designs
is a boon allowing systems and applications to evolve independently by establishing limitations on the
size of the design space where applications couple with storage. Programmable
storage lifts the veil on the system and, with it, forces developers of higher-level services
to confront a large and expanding set of possible designs.

In this section we elucidate the matter of design space size and complexity in
programmable storage. We report on our experience building and optimizing
\emph{multiple} functionally equivalent implementations of the CORFU protocol
in Ceph, demonstrating that static selection of optimization strategies and tuning
decisions can lead to performance portability challenges in programmable
storage systems.

\subsection{System Tunables and Hardware}

A recent version of Ceph (v10.2.0-1281-g1f03205) has 994 tunable parameters.
195 parameters pertain to the object server itself and 95 parameters focus on
low-level storage abstractions built on XFS or BlueStore. Ceph also has
tunables controlling the characteristics of underlying subsystems, including and RocksDB (5
tunables), journals (24 tunables).
Previous investigations exploring the application of 
auto-tuning~\cite{behzad:sc2013-autotuning} techniques to systems exhibiting a large space of parameters 
met with limited success. Challenges associated with the approach are exacerbated in the context of
application-specific modifications and dynamically changing workloads.

{\bf Hardware.} Ceph is designed to run on a wide variety of commodity
hardware as well as new NVMe devices. All these devices encompass specific sets of
characteristics and tunables (e.g., the IO operation scheduler type). In our
experiments, we tested SSD, HDDs, and NVMe devices and discovered a wide range of
behaviors and performance profiles. While we generally observe the expected
result of faster devices, choosing the best implementation strategy is highly dependent on hardware.
%The changes
%in Ceph required to fully exploit the performance profile of NVMe, persistent
%memory, and RDMA networks will likely result in new design trade-offs for
%application-specific interfaces.

Evolving hardware and system tunables presents a challenge in optimizing
systems, even in static cases with fixed workloads. Programmable storage
approaches that introduce application-specific interfaces that are sensitive
to changes in workloads and the cost models of low-level interfaces that are
subject to change  greatly increase the design space and set of concerns to be
addressed by programmers.

\textbf{Takeaway:} Evolving hardware and system tunables presents a challenge
in optimizing systems, even in static cases with fixed workloads. Programmable
storage approaches that introduce application-specific interfaces which are
sensitive to changes in workloads, and sensitive to the cost models of
low-level interfaces that are subject to change, greatly increase the design
space and set of concerns to be addressed by programmers.

%\textbf{Takeaway:} The ever-evolving space of hardware and system tunables presents a challenge
%in optimizing systems, even in static cases with fixed workloads. Programmable
%storage approaches introduce application-specific interfaces sensitive to changes in workloads and 
%guided by cost models integrating low-level interface performance considerations.
%In particular, the low-level performance considerations promise to significantly increase the design space and set of
%concerns to be addressed by programmers.

\subsection{Software}

The primary source of complexity in large storage systems is, unsurprisingly,
the vast amount of software written to handle challenges like fault-tolerance
and consistency in distributed heterogeneous environments. We have found that
even routine upgrades can cause performance regressions manifesting in obstacles 
for adopters of a programmable storage approach to development.

The CORFU protocol stripes a log across a cluster of storage
devices, where each device exposes a custom 64-bit write-once address space for reading
and writing log entries. While this interface can be built directly into flash
devices~\cite{wei:systor13}, we constructed four different versions in Ceph
each as a software abstraction over the existing object interface.
Each of our software-based implementations differ with respect to optimization
strategies utilizing internal system interfaces. For instance, one
implementation uses a key-value interface to manage the address space index
and entry data, while another implementation stores the entry data using a
byte-addressable interface. 

Figure~\ref{fig:phy-design} shows the append throughput of four such
implementations running on two versions of Ceph from 2014 and 2016. The data indicate 
performance in general is significantly better in the newer version of Ceph. 
However, the implementations manifest another interesting observation. The top two
implementations, run on a version of Ceph from 2014, perform with nearly identical throughput, 
but have strikingly different implementation complexities. The performance of the same
implementations on the newer version of Ceph illustrates a challenge: developers face a
reasonable choice of a simpler implementation in the 2014 version of Ceph and a storage interface
which will perform worse in the 2016 version of Ceph, requiring a significant overhaul of low-level
interface implementations.

\begin{figure*}[t]
    \centering
    \begin{subfigure}[b]{.3\linewidth}
        \centering
        \includegraphics[width=1.0\linewidth]{jewel_v_firefly_pd.png}
        \caption{}
        %\caption{Relative performance differences can be drastic after a software
        %upgrade of the underlying storage system.}
        \label{fig:phy-design}
    \end{subfigure}
    \begin{subfigure}[b]{.3\linewidth}
        \centering
        \includegraphics[width=1.0\linewidth]{batching.png}
        \caption{}
        %\caption{Total throughput with and without batching.}
        \label{fig:batching}
    \end{subfigure}
    \begin{subfigure}[b]{.3\linewidth}
        \centering
        \includegraphics[width=1.0\linewidth]{batching-outlier-detect.png}
        \caption{}
        %\caption{Identifying and handling an outlier independently maintains the
        %beneifts of batching without the performance degredation of unecessarily
        %large I/O requests.}
        \label{fig:batching-outlier}
    \end{subfigure}
    \caption{(a) relative performance differences are different after storage
    software upgrade. (b) total throughput with and without batching. (c)
    identifying and handling an outlier maintains the benefits
    of batching without the performance degradation of unnecessarily large I/O
    requests.}
\end{figure*}

\subsubsection{Application-specific Group Commit}
\label{sec:batch}

Group commit is a technique used in database query execution that combines
multiple transactions in order to amortize over per-transaction fixed costs
like logging. Figure~\ref{fig:batching} shows the performance impact of using
a \emph{group commit}-like technique for batching log appends from independent
clients into a single request. The \emph{Basic-Batch} case implements group
commit at the request level, but processes each sub-request (i.e. append)
independently using low-level I/O interfaces. The modest performance increase
is attributed to a reduction in average per-request costs related to network
round-trips. Compared with the \emph{Basic-Batch} case, the \emph{Opt-Batch}
implementation is able to achieve significantly higher performance by
constructing more efficient I/O requests using range queries and data sieving
techniques~\cite{750599} afforded by the low-level I/O interfaces.

The batched execution technique of group commit can significantly increase
throughput, but the story is much more complex. The ability to apply this
technique requires tuning parameters capable of triggering cascading performance impacts.
For example, adding artificial delays to increase batch size affects other metrics such as latency.
Additionally, while the performance impact of application-specific batching is significant,
techniques such as range queries and data sieving are sensitive to outliers
resulting from buggy or slow clients.

When outliers occur in a batch, naively building large I/O requests can result
in a large amount of wasted I/O. Figure~\ref{fig:batching-outlier} highlights
the scope of this challenge. The \emph{Basic-Batch} case handles each request
in a batch independently and, while the resulting performance is worse relative to the
other techniques, it is not sensitive to outliers. On the other hand, the \emph{Opt-Batch}
implementation achieves high append throughput, but performance degrades as
the magnitude of the outliers in the batch increases. In contrast, an
\emph{Outlier-Aware} policy applies a simple heuristic to identify and handle outliers 
independently, resulting in only a slight decrease in
performance over the best case.

\textbf{Takeaway:} Choosing the best implementation of a storage interface
depends on the timing of development (Ceph version), the expertise of
programmers and administrators (Ceph features), the tuning parameters and hardware
selection, as well as system-level and application-specific workload changes.
A direct consequence of such a large design space is forcing engineers
to duplicate efforts when hardware and software characteristics change. Such a 
circumstance necessitates the sub-optimal pursuit of unnecessary on-going work 
and increases the risk of introducing bugs that, in the best case, affect a single application and, in monolithic
designs, are more likely to cause systemic data loss.

We believe a better understanding of application and interface
semantics exposes a frontier of new and better approaches with more optimal maintenance requirements
than hard-coded and hand-tuned implementations. An ideal solution to these challenges is an automated system
search of \emph{implementations}---not simply tuning parameters---based on
programmer-produced specifications of storage interfaces in a process
independent of optimization strategies and guaranteed to not introduce
correctness bugs. Next we'll discuss a candidate approach using a declarative
language for interface specification.
